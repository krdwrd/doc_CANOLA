\begin{longversion}
%
%
Two fundamental ideas behind this part of the system are:
firstly, Web pages have a textual representation, namely the text they contain, a structural representation, namely their DOM tree, and a visual representation, namely their rendered view -- all representations should be considered when automatically cleaning Web pages, and consequently, all should be annotated during acquisition of training data for ML tasks.
Secondly, data acquisition for training of supervised ML algorithms should preserve pristine, unmodified versions of Web pages -- this will help to reproduce results \emph{and} to compare those of different architectures.

% What
\subsection{Functional Walk-Through}

%What, Who, How, Result
Gathering a set of sample pages is at the beginning of tagging new data. 
The process needs to be coordinated by the administrators of the system, i.e.~server level access is needed to make new corpora available for later tagging by users. 
The Process starts with a list of seed terms which are used to construct an ad-hoc corpus of Web pages where the result is a list of Uniform Resource Locators (URL\footnote{see \cite{URL} for details -- but also \cite{w3.org/Addressing}.}).  
 
%What, Who, How, Result
The URL list is then \emph{harvested}, i.e.~the according Web pages are downloaded and saved for further processing. 
This process is coordinated by the administrators of the system and is started as automated batch-job on the server where its input is the URL List and the result is the set of downloaded Web pages and their content.

%What, Who, How, Result
These Web Pages are then available online to users for tagging, i.e.~there are no constraints on who is able to access these pages; 
however, keeping track of \emph{who tagged what} requires to differentiate between users, and hence, registration with the system, viz.~logging in. 
The Web pages are accessible via the KrdWrd Add-on in combination\footnotemark~with the Web Services hosted on \cite[Web Site]{krdwrd.org}.
\footnotetext{Indeed, the data is accessible with \emph{any} browser -- but the KrdWrd Add-on enhances the experience.}

%What, Who, How, Result
Users can tag new, alter or redisplay formerly tagged Web pages with the help of the KrdWrd Add-on.
The KrdWrd Add-on builds upon and extends the functionality of the Firefox \cite{firefox} browser and facilitates the visual tagging of Web pages, i.e.~users are provided with an accurate Web page presentation and annotation utility in a typical browsing environment.
Readily (or partly) tagged pages are directly sent back to the server for storage in the KrdWrd Corpus data pool and for further processing.  

%What, Who, How, Result
Updated or newly submitted tagging results are regularly merged, i.e.~submitted results from different users for the same content are processed and compiled into a majority-driven uniform view.
This automated process uses a \emph{winner takes all strategy} and runs regularly on the server -- without further ado.
The \emph{merged} content is stored in the KrdWrd data pool and hence, available for browsing, viewing, and analysis by the KrdWrd Add-on\footnotemark[\value{footnote}] and furthermore, it can be used as training data for Machine Learning algorithms. 


% What
\subsection{Implementation Survey}

The KrdWrd Infrastructure consists of several components that bring along the overall functionality of the system. 
They are run either on the KrdWrd Server or are part of the KrdWrd Add-on and hence, build upon and extend the functionality of the Firefox browser.
The Server components are hosted on a Debian GNU/Linux \cite{debian.org} powered machine.
However, the requirements\footnote{These include sed, awk, python, bash, subversion, XULRunner, wwwoffle, apache, R.} are rather limited and many other standard linux - or linux-like - systems should easily suffice, and even other platforms should be able to host the system. 
Nevertheless, the KrdWrd Add-on strictly runs only as an extension of the Firefox browser, version 3\footnote{But it could be converted into a self-contained XULRunner application.}.

Access to the system is given as HTTP Service hosted on \url{krdwrd.org}, an SSL-certified virtual host running on an Apache Web Server \cite{httpd.apache.org} accompanied by mailing services, a dedicated trac as Wiki and issue tracking system for software development (extended with a mailing extension), and subversion \cite{subversion} as version control system.
The interfacing between the KrdWrd Add-on and the Web Server is done via CGI \cite{cgi} scripts, which itself are mostly written in the Python programming language \cite{python}.
%
%
\end{longversion}
